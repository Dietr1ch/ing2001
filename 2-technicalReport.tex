% Informe t�cnico (M�ximo 20 p�ginas)
% ===============
\section{Reporte t�cnico}

% Identificaci�n y contextualizaci�n
% ----------------------------------
%   Contextualiza, en no m�s de una plana, el lugar de pr�ctica
%         eg:
%            - Caracter�sticas de la empresa
%            - �rea
%            - Tama�o de la empresa en relaci�n al rubro
%            - etc
%\subsection{La empresa}
%\paragraph{}



% Generaci�n y justificaci�n
% --------------------------
%   Identifica el objetivo o inter�s de la empresa a ser logrado durante
%         la pr�ctica y plantea un problema o pregunta a ser resuelto.

%  Propone una metodolog�a, herramientas y/o modelos para el an�lisis
%         dise�o, desarrollo y soluci�n del objetivo/problema (utiliza supuestos
%         si no fueron aplicados durante el periodo de pr�ctica).

%  Describe los an�lisis, mediciones o aplicaciones y los softwares
%         necesarios para la resoluci�n del problema/demanda (utiliza supuestos
%         si no fueron aplicados durante el periodo de pr�ctica).

%  Describe los resultados obtenidos apoy�ndose en indicadores concretos
%         o mediciones reales o estimadas.
%\subsection{Los objetivos}
%\paragraph{}



% Conclusi�n
% ----------
%  Concluye los resultados, incluyendo: limitaciones y relevancia de los
%         an�lisis obtenidos; las implicancias futuras de las decisiones tomadas
%         y se discuten eventuales modificaciones y pasos a seguir.
%\subsection{Conclusi�n}
